% generated by GAPDoc2LaTeX from XML source (Frank Luebeck)
\documentclass[a4paper,11pt]{report}

\usepackage[top=37mm,bottom=37mm,left=27mm,right=27mm]{geometry}
\sloppy
\pagestyle{myheadings}
\usepackage{amssymb}
\usepackage[utf8]{inputenc}
\usepackage{makeidx}
\makeindex
\usepackage{color}
\definecolor{FireBrick}{rgb}{0.5812,0.0074,0.0083}
\definecolor{RoyalBlue}{rgb}{0.0236,0.0894,0.6179}
\definecolor{RoyalGreen}{rgb}{0.0236,0.6179,0.0894}
\definecolor{RoyalRed}{rgb}{0.6179,0.0236,0.0894}
\definecolor{LightBlue}{rgb}{0.8544,0.9511,1.0000}
\definecolor{Black}{rgb}{0.0,0.0,0.0}

\definecolor{linkColor}{rgb}{0.0,0.0,0.554}
\definecolor{citeColor}{rgb}{0.0,0.0,0.554}
\definecolor{fileColor}{rgb}{0.0,0.0,0.554}
\definecolor{urlColor}{rgb}{0.0,0.0,0.554}
\definecolor{promptColor}{rgb}{0.0,0.0,0.589}
\definecolor{brkpromptColor}{rgb}{0.589,0.0,0.0}
\definecolor{gapinputColor}{rgb}{0.589,0.0,0.0}
\definecolor{gapoutputColor}{rgb}{0.0,0.0,0.0}

%%  for a long time these were red and blue by default,
%%  now black, but keep variables to overwrite
\definecolor{FuncColor}{rgb}{0.0,0.0,0.0}
%% strange name because of pdflatex bug:
\definecolor{Chapter }{rgb}{0.0,0.0,0.0}
\definecolor{DarkOlive}{rgb}{0.1047,0.2412,0.0064}


\usepackage{fancyvrb}

\usepackage{mathptmx,helvet}
\usepackage[T1]{fontenc}
\usepackage{textcomp}


\usepackage[
            pdftex=true,
            bookmarks=true,        
            a4paper=true,
            pdftitle={Written with GAPDoc},
            pdfcreator={LaTeX with hyperref package / GAPDoc},
            colorlinks=true,
            backref=page,
            breaklinks=true,
            linkcolor=linkColor,
            citecolor=citeColor,
            filecolor=fileColor,
            urlcolor=urlColor,
            pdfpagemode={UseNone}, 
           ]{hyperref}

\newcommand{\maintitlesize}{\fontsize{50}{55}\selectfont}

% write page numbers to a .pnr log file for online help
\newwrite\pagenrlog
\immediate\openout\pagenrlog =\jobname.pnr
\immediate\write\pagenrlog{PAGENRS := [}
\newcommand{\logpage}[1]{\protect\write\pagenrlog{#1, \thepage,}}
%% were never documented, give conflicts with some additional packages

\newcommand{\GAP}{\textsf{GAP}}

%% nicer description environments, allows long labels
\usepackage{enumitem}
\setdescription{style=nextline}

%% depth of toc
\setcounter{tocdepth}{1}





%% command for ColorPrompt style examples
\newcommand{\gapprompt}[1]{\color{promptColor}{\bfseries #1}}
\newcommand{\gapbrkprompt}[1]{\color{brkpromptColor}{\bfseries #1}}
\newcommand{\gapinput}[1]{\color{gapinputColor}{#1}}


\begin{document}

\logpage{[ 0, 0, 0 ]}
\begin{titlepage}
\mbox{}\vfill

\begin{center}{\maintitlesize \textbf{ corefreesub \mbox{}}}\\
\vfill

\hypersetup{pdftitle= corefreesub }
\markright{\scriptsize \mbox{}\hfill  corefreesub  \hfill\mbox{}}
{\Huge \textbf{ A \textsf{GAP} Package for calculating the core-free subgroups and their faithful transitive
permutation representations \mbox{}}}\\
\vfill

{\Huge  0.3 \mbox{}}\\[1cm]
{ 7 August 2023 \mbox{}}\\[1cm]
\mbox{}\\[2cm]
{\Large \textbf{ Claudio Alexandre Piedade\\
   \mbox{}}}\\
{\Large \textbf{ Manuel Delgado\\
   \mbox{}}}\\
\hypersetup{pdfauthor= Claudio Alexandre Piedade\\
   ;  Manuel Delgado\\
   }
\end{center}\vfill

\mbox{}\\
{\mbox{}\\
\small \noindent \textbf{ Claudio Alexandre Piedade\\
   }  Email: \href{mailto://claudio.piedade@fc.up.pt} {\texttt{claudio.piedade@fc.up.pt}}\\
  Homepage: \href{https://www.fc.up.pt/pessoas/claudio.piedade/} {\texttt{https://www.fc.up.pt/pessoas/claudio.piedade/}}}\\
{\mbox{}\\
\small \noindent \textbf{ Manuel Delgado\\
   }  Email: \href{mailto://mdelgado@fc.up.pt} {\texttt{mdelgado@fc.up.pt}}\\
  Homepage: \href{https://cmup.fc.up.pt/cmup/mdelgado/} {\texttt{https://cmup.fc.up.pt/cmup/mdelgado/}}}\\
\end{titlepage}

\newpage\setcounter{page}{2}
{\small 
\section*{Copyright}
\logpage{[ 0, 0, 1 ]}
 \textsf{corefreesub} package is free software; you can redistribute it and/or modify it under the
terms of the \href{http://www.fsf.org/licenses/gpl.html} {GNU General Public License} as published by the Free Software Foundation; either version 2 of the License,
or (at your option) any later version. \mbox{}}\\[1cm]
{\small 
\section*{Acknowledgements}
\logpage{[ 0, 0, 2 ]}
 The authours wish to thank all the comments, suggestions and issue reporting
from users and developers of \textsf{GAP}, both past and future. Both authors were partially supported by CMUP, member
of LASI, which is financed by Portuguese national funds through FCT
{\textendash} Funda{\c c}{\~a}o para a Ci{\^e}ncia e a Tecnologia, I.P., under
the project with references UIDB/00144/2020 and UIDP/00144/2020. \mbox{}}\\[1cm]
\newpage

\def\contentsname{Contents\logpage{[ 0, 0, 3 ]}}

\tableofcontents
\newpage

     
\chapter{\textcolor{Chapter }{Introduction}}\label{Chapter_Introduction}
\logpage{[ 1, 0, 0 ]}
\hyperdef{L}{X7DFB63A97E67C0A1}{}
{
  

 The \textsf{corefreesub} package was created to calculate core-free subgroups of a group, their
indexes, and faithful transitive permutation representations. 

 A core-free subgroup of a group \mbox{\texttt{\mdseries\slshape G}} is a subgroup \mbox{\texttt{\mdseries\slshape H}} such that 
\[ \cap_{g\in G} H = \{id_G\}. \]
 These subgroups are important since the action of \mbox{\texttt{\mdseries\slshape G}} on the cosets of \mbox{\texttt{\mdseries\slshape H}} is both transitive and faithful. Hence, this gives us a faithful transitive
permutation representation of \mbox{\texttt{\mdseries\slshape G}} with degree \mbox{\texttt{\mdseries\slshape n}}, where \mbox{\texttt{\mdseries\slshape n}} is the index of \mbox{\texttt{\mdseries\slshape H}} in \mbox{\texttt{\mdseries\slshape G}}. 

 There are many articles studying faithful permutation representation of
groups, such as \cite{johnson_minimal_1971}, \cite{easdown_minimal_1988}, \cite{saunders_minimal_2014} and \cite{easdown_minimal_2016}. However the restriction on transitive actions is more recent and there are
fewer studies like \cite{FP20Tor},\cite{FP21Cor},\cite{FP21Hyper} and \cite{FP22Loc}. 

 During C.A. Piedade's PhD thesis, he studied many of these faithful transitive
permutation representations of automorphism groups of abstract regular
polytopes and hypertopes. It was also during this period that this author
noticed the abcense of functions/methods in GAP to compute core-free subgroups
of a group. As a consequence, he created many functions to help in his
research, resulting in many of the functions and methods implemented in this
package. 

 One of the important tools for studying faithful transitive permutation
representations is by using \mbox{\texttt{\mdseries\slshape faithful transitive permutation representation graphs}}, which are \mbox{\texttt{\mdseries\slshape Schreier coset graphs}}. A \mbox{\texttt{\mdseries\slshape Schreier coset graph}} is a graph associated with a group \mbox{\texttt{\mdseries\slshape G}}, its generators and a subgroup \mbox{\texttt{\mdseries\slshape H}} of \mbox{\texttt{\mdseries\slshape G}}. The vertices of the graph are the right cosets of \mbox{\texttt{\mdseries\slshape H}} and there is a directed edge $(Hx,Hy)$ with label $g$ if $g$ is a generator of \mbox{\texttt{\mdseries\slshape G}} and $Hxg = Hy$. When $g$ is an involution, the two directed edges $(Hx, Hy)$ and $(Hy, Hx)$ are replaced by a single undirected edge $\{Hx, Hy\}$ with label $g$. 

 In the \textsf{corefreesub} package, this can achieved by creating graphs as DOT files and using an
adaptation of the visualization package developed by M. Delgado et al. \cite{IntPic} \cite{Automata}, which can be found on Chapter 4. 

 This package was created using the GAP Package \mbox{\texttt{\mdseries\slshape PackageMaker}} \cite{PackageMaker}, with documentation done using \mbox{\texttt{\mdseries\slshape AutoDoc}} \cite{AutoDoc} 

 
\section{\textcolor{Chapter }{Instalation}}\label{Chapter_Introduction_Section_Instalation}
\logpage{[ 1, 1, 0 ]}
\hyperdef{L}{X7A5BA34E7CC6300F}{}
{
  

 To install this package, you can simply copy the folder of \textsf{corefreesub} and its contents into your \mbox{\texttt{\mdseries\slshape /pkg}} folder inside your \textsf{GAP} instalation folder. This should work for Windows, Ubuntu and MacOS. If you are
using GAP.app on MacOS, the \textsf{corefreesub} folder should be copied into your user Library/Preferences/GAP/pkg folder. 

 This package was tested with \textsf{GAP} version greater or equal to 4.11. 

 }

 
\section{\textcolor{Chapter }{Testing your instalation}}\label{Chapter_Introduction_Section_Testing_your_instalation}
\logpage{[ 1, 2, 0 ]}
\hyperdef{L}{X8541922979222078}{}
{
  

 To test your instalation, you can run the function \mbox{\texttt{\mdseries\slshape CF{\textunderscore}TESTALL()}}. This function will run two sets of tests, one dependent on the documentation
of the \textsf{corefreesub} package and another with assertions with groups with bigger size. 

 If the test runs with no issue, the output should look something similar to
the following: 
\begin{Verbatim}[commandchars=!@|,fontsize=\small,frame=single,label=Example]
  !gapprompt@gap>| !gapinput@CF_TESTALL();|
  Running list 1 . . .
  gap>
\end{Verbatim}
 This tests will also produce two pictures that are supposed to be outputed and
open in the user system. If the tests run with no error but they do not output
any of the graphs, then it may mean the user might not be able to use this
functionality. If so, please report an issue on \href{https://github.com/CAPiedade/corefreesub/issues} {CoreFreeSub GitHub Issues}. 

 }

 }

   
\chapter{\textcolor{Chapter }{Obtaining Core-Free Subgroups}}\label{Chapter_Obtaining_Core-Free_Subgroups}
\logpage{[ 2, 0, 0 ]}
\hyperdef{L}{X78D8803A84DB6CE8}{}
{
  
\section{\textcolor{Chapter }{Core-Free Subgroups}}\label{Chapter_Obtaining_Core-Free_Subgroups_Section_Core-Free_Subgroups}
\logpage{[ 2, 1, 0 ]}
\hyperdef{L}{X79EF503F837B7702}{}
{
  

 A core-free subgroup is a subgroup in which its (normal) core is trivial. 

 

\subsection{\textcolor{Chapter }{IsCoreFree}}
\logpage{[ 2, 1, 1 ]}\nobreak
\hyperdef{L}{X7A9042C27F0BA85B}{}
{\noindent\textcolor{FuncColor}{$\triangleright$\enspace\texttt{IsCoreFree({\mdseries\slshape G, H})\index{IsCoreFree@\texttt{IsCoreFree}}
\label{IsCoreFree}
}\hfill{\scriptsize (function)}}\\
\textbf{\indent Returns:\ }
a boolean 



 Given a group \mbox{\texttt{\mdseries\slshape G}} and one of its subgroups \mbox{\texttt{\mdseries\slshape H}}, it returns whether \mbox{\texttt{\mdseries\slshape H}} is core-free in \mbox{\texttt{\mdseries\slshape G}}. 
\begin{Verbatim}[commandchars=!@|,fontsize=\small,frame=single,label=Example]
  !gapprompt@gap>| !gapinput@G := SymmetricGroup(4);; H := Subgroup(G, [(1,3)(2,4)]);;|
  !gapprompt@gap>| !gapinput@Core(G,H);|
  Group(())
  !gapprompt@gap>| !gapinput@IsCoreFree(G,H);|
  true
  !gapprompt@gap>| !gapinput@H := Subgroup(G, [(1,4)(2,3), (1,3)(2,4)]);;|
  !gapprompt@gap>| !gapinput@IsCoreFree(G,H);|
  false
  !gapprompt@gap>| !gapinput@Core(G,H);# H is a normal subgroup of G, hence it does not have a trivial core|
  Group([ (1,4)(2,3), (1,3)(2,4) ])
\end{Verbatim}
 }

 

 

\subsection{\textcolor{Chapter }{CoreFreeConjugacyClassesSubgroups}}
\logpage{[ 2, 1, 2 ]}\nobreak
\hyperdef{L}{X7D0391668782765C}{}
{\noindent\textcolor{FuncColor}{$\triangleright$\enspace\texttt{CoreFreeConjugacyClassesSubgroups({\mdseries\slshape G})\index{CoreFreeConjugacyClassesSubgroups@\texttt{CoreFreeConjugacyClassesSubgroups}}
\label{CoreFreeConjugacyClassesSubgroups}
}\hfill{\scriptsize (function)}}\\
\textbf{\indent Returns:\ }
a list 



 Returns a list of all conjugacy classes of core-free subgroups of \mbox{\texttt{\mdseries\slshape G}} 
\begin{Verbatim}[commandchars=!@|,fontsize=\small,frame=single,label=Example]
  !gapprompt@gap>| !gapinput@G := SymmetricGroup(4);; dh := DihedralGroup(10);;|
  !gapprompt@gap>| !gapinput@CoreFreeConjugacyClassesSubgroups(G);|
  [ Group( () )^G, Group( [ (1,3)(2,4) ] )^G, Group( [ (3,4) ] )^G,
  Group( [ (2,4,3) ] )^G, Group( [ (3,4), (1,2)(3,4) ] )^G,
  Group( [ (1,3,2,4), (1,2)(3,4) ] )^G, Group( [ (3,4), (2,4,3) ] )^G ]
  !gapprompt@gap>| !gapinput@CoreFreeConjugacyClassesSubgroups(dh);|
  [ Group( <identity> of ... )^G, Group( [ f1 ] )^G ] 
\end{Verbatim}
 }

 

 

\subsection{\textcolor{Chapter }{AllCoreFreeSubgroups}}
\logpage{[ 2, 1, 3 ]}\nobreak
\hyperdef{L}{X821F9BD687383C1F}{}
{\noindent\textcolor{FuncColor}{$\triangleright$\enspace\texttt{AllCoreFreeSubgroups({\mdseries\slshape G})\index{AllCoreFreeSubgroups@\texttt{AllCoreFreeSubgroups}}
\label{AllCoreFreeSubgroups}
}\hfill{\scriptsize (function)}}\\
\textbf{\indent Returns:\ }
a list 



 Returns a list of all core-free subgroups of \mbox{\texttt{\mdseries\slshape G}} 
\begin{Verbatim}[commandchars=!@|,fontsize=\small,frame=single,label=Example]
  !gapprompt@gap>| !gapinput@G := SymmetricGroup(4);; dh := DihedralGroup(10);;|
  !gapprompt@gap>| !gapinput@AllCoreFreeSubgroups(G);|
  [ Group(()), Group([ (1,3)(2,4) ]), Group([ (1,4)(2,3) ]), Group([ (1,2)(3,4) ]),
   Group([ (3,4) ]), Group([ (2,4) ]), Group([ (2,3) ]), Group([ (1,4) ]),
   Group([ (1,3) ]), Group([ (1,2) ]), Group([ (2,4,3) ]), Group([ (1,3,2) ]),
   Group([ (1,3,4) ]), Group([ (1,4,2) ]), Group([ (3,4), (1,2)(3,4) ]),
   Group([ (2,4), (1,3)(2,4) ]), Group([ (2,3), (1,4)(2,3) ]),
   Group([ (1,3,2,4), (1,2)(3,4) ]), Group([ (1,2,3,4), (1,3)(2,4) ]),
   Group([ (1,2,4,3), (1,4)(2,3) ]), Group([ (3,4), (2,4,3) ]),
   Group([ (1,3), (1,3,2) ]), Group([ (1,3), (1,3,4) ]), Group([ (1,4), (1,4,2) ])
  ]
  !gapprompt@gap>| !gapinput@AllCoreFreeSubgroups(dh);|
  [ Group([  ]), Group([ f1 ]), Group([ f1*f2^2 ]), Group([ f1*f2^4 ]), 
  Group([ f1*f2 ]), Group([ f1*f2^3 ]) ]
\end{Verbatim}
 }

 }

 
\section{\textcolor{Chapter }{Degrees of Core-Free subgroups}}\label{Chapter_Obtaining_Core-Free_Subgroups_Section_Degrees_of_Core-Free_subgroups}
\logpage{[ 2, 2, 0 ]}
\hyperdef{L}{X847B222E84342E88}{}
{
  

 

\subsection{\textcolor{Chapter }{CoreFreeDegrees}}
\logpage{[ 2, 2, 1 ]}\nobreak
\hyperdef{L}{X844F04457C6EDC67}{}
{\noindent\textcolor{FuncColor}{$\triangleright$\enspace\texttt{CoreFreeDegrees({\mdseries\slshape G})\index{CoreFreeDegrees@\texttt{CoreFreeDegrees}}
\label{CoreFreeDegrees}
}\hfill{\scriptsize (function)}}\\
\textbf{\indent Returns:\ }
a list 



 Returns a list of all possible degrees of faithful transitive permutation
representations of \mbox{\texttt{\mdseries\slshape G}}. The degrees of a faithful transitive permutation representation of \mbox{\texttt{\mdseries\slshape G}} are the index of its core-free subgroups. 
\begin{Verbatim}[commandchars=!@|,fontsize=\small,frame=single,label=Example]
  !gapprompt@gap>| !gapinput@G := SymmetricGroup(4);; dh := DihedralGroup(10);;|
  !gapprompt@gap>| !gapinput@CoreFreeDegrees(G);|
  [ 24, 12, 8, 6, 4 ]
  !gapprompt@gap>| !gapinput@CoreFreeDegrees(dh);|
  [10, 5]
\end{Verbatim}
 }

 }

 }

   
\chapter{\textcolor{Chapter }{Faithful Transitive Permutation Representations}}\label{Chapter_Faithful_Transitive_Permutation_Representations}
\logpage{[ 3, 0, 0 ]}
\hyperdef{L}{X857DD6BF86E302D0}{}
{
  

 The action of a group G on the coset space of a subgroup gives us a transitive
permutation representation of the group. Whenever the subgroup is core-free,
we have that the action of G on the coset space of the subgroup will be
faithful. Moreover, the stabilizer of a point on a faithful transitive
permutation representation of G will always be a core-free subgroup. 

 
\section{\textcolor{Chapter }{Obtaining Faithful Transitive Permutation Representations}}\label{Chapter_Faithful_Transitive_Permutation_Representations_Section_Obtaining_Faithful_Transitive_Permutation_Representations}
\logpage{[ 3, 1, 0 ]}
\hyperdef{L}{X839E0E8B87479AB5}{}
{
  

 

\subsection{\textcolor{Chapter }{FaithfulTransitivePermutationRepresentations (for IsGroup)}}
\logpage{[ 3, 1, 1 ]}\nobreak
\hyperdef{L}{X80D610507B1029C9}{}
{\noindent\textcolor{FuncColor}{$\triangleright$\enspace\texttt{FaithfulTransitivePermutationRepresentations({\mdseries\slshape G[, all{\textunderscore}ftpr]})\index{FaithfulTransitivePermutationRepresentations@\texttt{Faithful}\-\texttt{Transitive}\-\texttt{Permutation}\-\texttt{Representations}!for IsGroup}
\label{FaithfulTransitivePermutationRepresentations:for IsGroup}
}\hfill{\scriptsize (operation)}}\\
\textbf{\indent Returns:\ }
a list 



 For a finite group \mbox{\texttt{\mdseries\slshape G}}, \mbox{\texttt{\mdseries\slshape FaithfulTransitivePermutationRepresentations}} returns a list of a faithful transitive permutation representation of \mbox{\texttt{\mdseries\slshape G}} for each degree. If \mbox{\texttt{\mdseries\slshape all{\textunderscore}ftpr}} is true, then it will return a list of all faithful transitive permutation
representations. 
\begin{Verbatim}[commandchars=!@|,fontsize=\small,frame=single,label=Example]
  !gapprompt@gap>| !gapinput@sp := SymplecticGroup(4,2);;|
  !gapprompt@gap>| !gapinput@CoreFreeDegrees(sp);|
  [ 720, 360, 240, 180, 144, 120, 90, 80, 72, 60, 45, 40, 36, 30, 20, 15, 12,
  10, 6 ]
  !gapprompt@gap>| !gapinput@ftprs := FaithfulTransitivePermutationRepresentations(sp);; |
  !gapprompt@gap>| !gapinput@Size(ftprs);|
  19
  !gapprompt@gap>| !gapinput@all_ftprs := FaithfulTransitivePermutationRepresentations(sp,true);; |
  !gapprompt@gap>| !gapinput@Size(all_ftprs);|
  54
\end{Verbatim}
 }

 }

 
\section{\textcolor{Chapter }{Faithful Transitive Permutation Representation of Minimal Degree}}\label{Chapter_Faithful_Transitive_Permutation_Representations_Section_Faithful_Transitive_Permutation_Representation_of_Minimal_Degree}
\logpage{[ 3, 2, 0 ]}
\hyperdef{L}{X836069DB849BD28D}{}
{
  

 To complement the already existing functions in GAP \mbox{\texttt{\mdseries\slshape MinimalFaithfulPermutationDegree}} and \mbox{\texttt{\mdseries\slshape MinimalFaithfulPermutationRepresentation}}, the following functions to retreive the \mbox{\texttt{\mdseries\slshape MinimalFaithfulTransitivePermutationRepresentation}} and \mbox{\texttt{\mdseries\slshape MinimalFaithfulTransitivePermutationDegree}}. 

 

\subsection{\textcolor{Chapter }{MinimalFaithfulTransitivePermutationRepresentation (for IsGroup)}}
\logpage{[ 3, 2, 1 ]}\nobreak
\hyperdef{L}{X8746DBA2866336CC}{}
{\noindent\textcolor{FuncColor}{$\triangleright$\enspace\texttt{MinimalFaithfulTransitivePermutationRepresentation({\mdseries\slshape G[, all{\textunderscore}minimal{\textunderscore}ftpr]})\index{MinimalFaithfulTransitivePermutationRepresentation@\texttt{Minimal}\-\texttt{Faithful}\-\texttt{Transitive}\-\texttt{Permutation}\-\texttt{Representation}!for IsGroup}
\label{MinimalFaithfulTransitivePermutationRepresentation:for IsGroup}
}\hfill{\scriptsize (operation)}}\\
\textbf{\indent Returns:\ }
an isomorphism (or a list of isomorphisms) 



 For a finite group \mbox{\texttt{\mdseries\slshape G}}, \mbox{\texttt{\mdseries\slshape MinimalFaithfulTransitivePermutationRepresentation}} returns an isomorphism of \mbox{\texttt{\mdseries\slshape G}} into the symmetric group of minimal degree acting transitively on its domain.
If \mbox{\texttt{\mdseries\slshape all{\textunderscore}minimal{\textunderscore}ftpr}} is set as \mbox{\texttt{\mdseries\slshape true}}, then it returns a list of all isomorphisms \mbox{\texttt{\mdseries\slshape G}} into the symmetric group of minimal degree. 
\begin{Verbatim}[commandchars=!@|,fontsize=\small,frame=single,label=Example]
  !gapprompt@gap>| !gapinput@sp := SymplecticGroup(4,2);;|
  !gapprompt@gap>| !gapinput@min_ftpr := MinimalFaithfulTransitivePermutationRepresentation(sp);|
  CompositionMapping( <action epimorphism>, <action isomorphism> )
  !gapprompt@gap>| !gapinput@min_ftpr(sp);|
  Group([ (1,6,4,3), (1,3)(2,4,6,5) ])
  !gapprompt@gap>| !gapinput@min_ftprs := MinimalFaithfulTransitivePermutationRepresentation(sp,true);|
  [ CompositionMapping( <action epimorphism>, <action isomorphism> ), 
  CompositionMapping( <action epimorphism>, <action isomorphism> ) ]
  !gapprompt@gap>| !gapinput@min_ftprs[2](sp);|
  Group([ (2,3,6,5), (1,3)(2,5,6,4) ])
\end{Verbatim}
 }

 

\subsection{\textcolor{Chapter }{MinimalFaithfulTransitivePermutationDegree}}
\logpage{[ 3, 2, 2 ]}\nobreak
\hyperdef{L}{X7B0EBFFA86B38A0F}{}
{\noindent\textcolor{FuncColor}{$\triangleright$\enspace\texttt{MinimalFaithfulTransitivePermutationDegree({\mdseries\slshape G})\index{MinimalFaithfulTransitivePermutationDegree@\texttt{Minimal}\-\texttt{Faithful}\-\texttt{Transitive}\-\texttt{Permutation}\-\texttt{Degree}}
\label{MinimalFaithfulTransitivePermutationDegree}
}\hfill{\scriptsize (function)}}\\
\textbf{\indent Returns:\ }
an integer 



 For a finite group \mbox{\texttt{\mdseries\slshape G}}, \mbox{\texttt{\mdseries\slshape MinimalFaithfulTransitivePermutationDegree}} returns the least positive integer n such that \mbox{\texttt{\mdseries\slshape G}} is isomorphic to a subgroup of the symmetric group of degree n acting
transitively on its domain. 
\begin{Verbatim}[commandchars=!@|,fontsize=\small,frame=single,label=Example]
  !gapprompt@gap>| !gapinput@sp := SymplecticGroup(4,2);; g:=SimpleGroup("PSL",3,5);;|
  !gapprompt@gap>| !gapinput@MinimalFaithfulTransitivePermutationDegree(sp);|
  6
  !gapprompt@gap>| !gapinput@MinimalFaithfulTransitivePermutationDegree(g);|
  31
\end{Verbatim}
 }

 }

 }

   
\chapter{\textcolor{Chapter }{Drawing the Faithful Transitive Permutation Representation Graph}}\label{Chapter_Drawing_the_Faithful_Transitive_Permutation_Representation_Graph}
\logpage{[ 4, 0, 0 ]}
\hyperdef{L}{X85BCED3F87DE1C07}{}
{
  
\section{\textcolor{Chapter }{Drawing functions}}\label{Chapter_Drawing_the_Faithful_Transitive_Permutation_Representation_Graph_Section_Drawing_functions}
\logpage{[ 4, 1, 0 ]}
\hyperdef{L}{X82F35AB4823114BD}{}
{
  

 One of the advantages of Faithful Transitive Permutation Representation Graph
are on Groups generated by involutions, such as C-groups. These graphs are
very useful in the research of abstract polytopes and hypertopes, mainly
called as "Schreier coset graphs" or "CPR graphs" in this area. Here we will
give a function that builds this graph given a permutation group generated by
involutions, a group and one of its core-free subgroups or by giving an
isomorphism of the group into the symmetric group acting faithfully and
transitively on its domain. To use Graphviz in order to create the image file,
you need to be running GAP on a Linux Environment (Windows Subsystem for Linux
is supported), with graphviz installed. 

 

\subsection{\textcolor{Chapter }{DotFTPRGraph (for IsPermGroup)}}
\logpage{[ 4, 1, 1 ]}\nobreak
\label{AutoDoc_generated_group1}
\hyperdef{L}{X785A165E8417A1D8}{}
{\noindent\textcolor{FuncColor}{$\triangleright$\enspace\texttt{DotFTPRGraph({\mdseries\slshape G})\index{DotFTPRGraph@\texttt{DotFTPRGraph}!for IsPermGroup}
\label{DotFTPRGraph:for IsPermGroup}
}\hfill{\scriptsize (operation)}}\\
\noindent\textcolor{FuncColor}{$\triangleright$\enspace\texttt{DotFTPRGraph({\mdseries\slshape G[, generators{\textunderscore}name]})\index{DotFTPRGraph@\texttt{DotFTPRGraph}!for IsPermGroup, IsList}
\label{DotFTPRGraph:for IsPermGroup, IsList}
}\hfill{\scriptsize (operation)}}\\
\noindent\textcolor{FuncColor}{$\triangleright$\enspace\texttt{DotFTPRGraph({\mdseries\slshape map})\index{DotFTPRGraph@\texttt{DotFTPRGraph}!for IsGeneralMapping}
\label{DotFTPRGraph:for IsGeneralMapping}
}\hfill{\scriptsize (operation)}}\\
\noindent\textcolor{FuncColor}{$\triangleright$\enspace\texttt{DotFTPRGraph({\mdseries\slshape map[, generators{\textunderscore}name]})\index{DotFTPRGraph@\texttt{DotFTPRGraph}!for IsGeneralMapping, IsList}
\label{DotFTPRGraph:for IsGeneralMapping, IsList}
}\hfill{\scriptsize (operation)}}\\
\noindent\textcolor{FuncColor}{$\triangleright$\enspace\texttt{DotFTPRGraph({\mdseries\slshape H, K})\index{DotFTPRGraph@\texttt{DotFTPRGraph}!for IsGroup,IsGroup}
\label{DotFTPRGraph:for IsGroup,IsGroup}
}\hfill{\scriptsize (operation)}}\\
\noindent\textcolor{FuncColor}{$\triangleright$\enspace\texttt{DotFTPRGraph({\mdseries\slshape H, K[, generators{\textunderscore}name]})\index{DotFTPRGraph@\texttt{DotFTPRGraph}!for IsGroup,IsGroup, IsList}
\label{DotFTPRGraph:for IsGroup,IsGroup, IsList}
}\hfill{\scriptsize (operation)}}\\
\textbf{\indent Returns:\ }
a graph written in dot 



 Given a transitive permutation group \mbox{\texttt{\mdseries\slshape G}}, a faithful transitive permutation representation of a group \mbox{\texttt{\mdseries\slshape map}} or a group \mbox{\texttt{\mdseries\slshape H}} and one of its core-free subgroups \mbox{\texttt{\mdseries\slshape K}}, the function will output the permutation representation graph written in the
language of a Dot file. If given a list of the name of the generators \mbox{\texttt{\mdseries\slshape generators{\textunderscore}name}}, these will be given to the label of their action on the graph. Otherwise,
the labels will be \mbox{\texttt{\mdseries\slshape r0, r1, r2, ...}} for the generators \mbox{\texttt{\mdseries\slshape G.1, G.2, G.3, ...}}. 

 
\begin{Verbatim}[commandchars=!@|,fontsize=\small,frame=single,label=Example]
  !gapprompt@gap>| !gapinput@G:= SymmetricGroup(4);;H:= Subgroup(G,[(1,2)]);;K:= Subgroup(G,[(1,2,3)]);;|
  !gapprompt@gap>| !gapinput@DotFTPRGraph(G);|
  "digraph {\n1 -> 2 [label = r1];\n 2 -> 3 [label = r1];\n 3 -> 4 [labe\
  l = r1];\n 4 -> 1 [label = r1];\n 1 -> 2 [label = r2,dir=none];\n }\n"
  !gapprompt@gap>| !gapinput@DotFTPRGraph(G,H);|
  "digraph {\n1 -> 10 [label = r1];\n 2 -> 12 [label = r1];\n 3 -> 11 [l\
  abel = r1];\n 4 -> 8 [label = r1];\n 5 -> 9 [label = r1];\n 6 -> 7 [la\
  bel = r1];\n 7 -> 3 [label = r1];\n 8 -> 2 [label = r1];\n 9 -> 1 [lab\
  el = r1];\n 10 -> 5 [label = r1];\n 11 -> 6 [label = r1];\n 12 -> 4 [l\
  abel = r1];\n 2 -> 3 [label = r2,dir=none];\n 4 -> 7 [label = r2,dir=n\
  one];\n 5 -> 8 [label = r2,dir=none];\n 6 -> 9 [label = r2,dir=none];\
  \n 10 -> 11 [label = r2,dir=none];\n }\n"
  !gapprompt@gap>| !gapinput@Print(DotFTPRGraph(FactorCosetAction(G,K),["A","B"]));|
  digraph {
  1 -> 3 [label = A];
  2 -> 4 [label = A];
  3 -> 5 [label = A];
  4 -> 6 [label = A];
  5 -> 8 [label = A];
  6 -> 7 [label = A];
  7 -> 2 [label = A];
  8 -> 1 [label = A];
  1 -> 2 [label = B,dir=none];
  3 -> 5 [label = B,dir=none];
  4 -> 6 [label = B,dir=none];
  7 -> 8 [label = B,dir=none];
  }
\end{Verbatim}
 

 }

 

\subsection{\textcolor{Chapter }{DrawFTPRGraph}}
\logpage{[ 4, 1, 2 ]}\nobreak
\hyperdef{L}{X8692BA897CFF04D4}{}
{\noindent\textcolor{FuncColor}{$\triangleright$\enspace\texttt{DrawFTPRGraph({\mdseries\slshape arg})\index{DrawFTPRGraph@\texttt{DrawFTPRGraph}}
\label{DrawFTPRGraph}
}\hfill{\scriptsize (function)}}\\
\textbf{\indent Returns:\ }
an image of the faithful transitive permutation representation graph 



 This global function takes as input the following arguments: 
\begin{itemize}
\item  \mbox{\texttt{\mdseries\slshape arg}} := \mbox{\texttt{\mdseries\slshape dotstring}}[, \mbox{\texttt{\mdseries\slshape rec}}] 
\item  \mbox{\texttt{\mdseries\slshape arg}} := \mbox{\texttt{\mdseries\slshape G}}[, \mbox{\texttt{\mdseries\slshape rec}}] 
\item  \mbox{\texttt{\mdseries\slshape arg}} := \mbox{\texttt{\mdseries\slshape map}}[,\mbox{\texttt{\mdseries\slshape rec}}] 
\item  \mbox{\texttt{\mdseries\slshape arg}} := \mbox{\texttt{\mdseries\slshape H}},\mbox{\texttt{\mdseries\slshape K}}[,\mbox{\texttt{\mdseries\slshape rec}}] 
\end{itemize}
 Given a string of a graph in dot \mbox{\texttt{\mdseries\slshape dotstring}}, this function will output and show an image of the graph. Alternatively, a
transitive permutation group \mbox{\texttt{\mdseries\slshape G}}, a faithful transitive permutation representation of a group \mbox{\texttt{\mdseries\slshape map}} or a group \mbox{\texttt{\mdseries\slshape H}} and one of its core-free subgroups \mbox{\texttt{\mdseries\slshape K}}, can be given. This will use \mbox{\texttt{\mdseries\slshape DotFTPRGraph}} to calculate the \mbox{\texttt{\mdseries\slshape dotstring}}. Moreover, extra parameters can be given as a form of a record \mbox{\texttt{\mdseries\slshape rec}}. The set of parameters that can be given inside a record can be found below,
with information regarding their effect: 
\begin{itemize}
\item  \mbox{\texttt{\mdseries\slshape layout}} - (a string) the engine that is used to calculate the layout of the vertices
and edges of graph to output in the dot image (not used for TeX output). The
supported layouts are "dot", "neato", "twopi", "circo", "fdp", "sfdp",
"patchwork", "osage". By default "neato" is used. 
\item  \mbox{\texttt{\mdseries\slshape directory}} - (a string) the name of the folder where the dot and image files are created.
By default, a temporary folder of GAP is used. 
\item  \mbox{\texttt{\mdseries\slshape path}} - (a string) the path where the directory will be created. If the directory is
not specified, a folder "tmp.viz" will be created at the determined path. If
no path is given, the default path is "\texttt{\symbol{126}}/". If no path nor
directory is given, it will be saved in a temporary path of GAP. 
\item  \mbox{\texttt{\mdseries\slshape file}} - (a string) the name of the dot and image files created. By default, the name
will be "vizpicture". 
\item  \mbox{\texttt{\mdseries\slshape filetype}} - (a string) the image file type that will be created. By default, the
filetype will be "pdf". 
\item  \mbox{\texttt{\mdseries\slshape viewer}} - (a string) the name of the visualizer used to open the image. The supported
ones are "evince","xpdf","xdg-open","okular", "gv", "open" (for the different
System Architectures). 
\item  \mbox{\texttt{\mdseries\slshape tikz}} - (a boolean) if true, then the function will produce a TeX file, compile it
to pdf and open. 
\item  \mbox{\texttt{\mdseries\slshape viewtexfile}} - (a boolean) if true, then the function will produce a TeX file and return
the text of the Tex file (but it will not compile and open any pdf from the
TeX file). 
\end{itemize}
 

 
\begin{Verbatim}[commandchars=!@|,fontsize=\small,frame=single,label=Example]
  !gapprompt@gap>| !gapinput@G:= SymmetricGroup(4);;H:= Subgroup(G,[(1,2)]);;K:= Subgroup(G,[(1,2,3)]);;|
  !gapprompt@gap>| !gapinput@DrawFTPRGraph(G);|
  !gapprompt@gap>| !gapinput@texfile := DrawFTPRGraph(G,H,rec(viewtexfile := true));; |
  !gapprompt@gap>| !gapinput@Print(texfile{[1..152]});|
  \documentclass{article}
  \usepackage[x11names, svgnames, rgb]{xcolor}
  \usepackage[utf8]{inputenc}
  \usepackage{tikz}
  \usetikzlibrary{snakes,arrows,shapes}
  !gapprompt@gap>| !gapinput@SetInfoLevel(InfoDrawFTPR,2);|
  !gapprompt@gap>| !gapinput@DrawFTPRGraph(FactorCosetAction(G,K),rec(directory:="myfolder",layout:="fdp"));|
  #I  The directory used is: ~/myfolder/
  
\end{Verbatim}
 }

 

\subsection{\textcolor{Chapter }{TeXFTPRGraph}}
\logpage{[ 4, 1, 3 ]}\nobreak
\hyperdef{L}{X7D0B40427CC8BFC5}{}
{\noindent\textcolor{FuncColor}{$\triangleright$\enspace\texttt{TeXFTPRGraph({\mdseries\slshape arg})\index{TeXFTPRGraph@\texttt{TeXFTPRGraph}}
\label{TeXFTPRGraph}
}\hfill{\scriptsize (function)}}\\
\textbf{\indent Returns:\ }
an image of the faithful transitive permutation representation graph 



 The same as \mbox{\texttt{\mdseries\slshape DrawFTPRGraph}} with the parameter \mbox{\texttt{\mdseries\slshape viewtexfile := true}}. }

 

\subsection{\textcolor{Chapter }{DrawTeXFTPRGraph}}
\logpage{[ 4, 1, 4 ]}\nobreak
\hyperdef{L}{X8721FD6E83582BA3}{}
{\noindent\textcolor{FuncColor}{$\triangleright$\enspace\texttt{DrawTeXFTPRGraph({\mdseries\slshape arg})\index{DrawTeXFTPRGraph@\texttt{DrawTeXFTPRGraph}}
\label{DrawTeXFTPRGraph}
}\hfill{\scriptsize (function)}}\\
\textbf{\indent Returns:\ }
an image of the faithful transitive permutation representation graph 



 The same as \mbox{\texttt{\mdseries\slshape DrawFTPRGraph}} with the parameter \mbox{\texttt{\mdseries\slshape tikz := true}}. }

 

 }

 
\section{\textcolor{Chapter }{Information Level of Drawing Functions}}\label{Chapter_Drawing_the_Faithful_Transitive_Permutation_Representation_Graph_Section_Information_Level_of_Drawing_Functions}
\logpage{[ 4, 2, 0 ]}
\hyperdef{L}{X8099E0467907387C}{}
{
  We can set the amount of verbosity of the functions "DrawFTPRGraph",
"TeXFTPRGraph" and "DrawTeXFTPRGraph", which can be controlled by the \mbox{\texttt{\mdseries\slshape InfoDrawFTPR}} variable. As of right now, there are only two levels of the \mbox{\texttt{\mdseries\slshape InfoDrawFTPR}} and, by default, the level is set as 1. To change to level 2, you can do the
following: 
\begin{Verbatim}[commandchars=!@|,fontsize=\small,frame=single,label=Example]
  !gapprompt@gap>| !gapinput@SetInfoLevel(InfoDrawFTPR,2);|
\end{Verbatim}
 Particularly, \mbox{\texttt{\mdseries\slshape InfoDrawFTPR}} in level 2 will give information regarding the location in which the files are
being created and processed. }

 }

 \def\bibname{References\logpage{[ "Bib", 0, 0 ]}
\hyperdef{L}{X7A6F98FD85F02BFE}{}
}

\bibliographystyle{alpha}
\bibliography{corefreesub.bib}

\addcontentsline{toc}{chapter}{References}

\def\indexname{Index\logpage{[ "Ind", 0, 0 ]}
\hyperdef{L}{X83A0356F839C696F}{}
}

\cleardoublepage
\phantomsection
\addcontentsline{toc}{chapter}{Index}


\printindex

\immediate\write\pagenrlog{["Ind", 0, 0], \arabic{page},}
\newpage
\immediate\write\pagenrlog{["End"], \arabic{page}];}
\immediate\closeout\pagenrlog
\end{document}
