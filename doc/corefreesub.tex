% generated by GAPDoc2LaTeX from XML source (Frank Luebeck)
\documentclass[a4paper,11pt]{report}

\usepackage[top=37mm,bottom=37mm,left=27mm,right=27mm]{geometry}
\sloppy
\pagestyle{myheadings}
\usepackage{amssymb}
\usepackage[utf8]{inputenc}
\usepackage{makeidx}
\makeindex
\usepackage{color}
\definecolor{FireBrick}{rgb}{0.5812,0.0074,0.0083}
\definecolor{RoyalBlue}{rgb}{0.0236,0.0894,0.6179}
\definecolor{RoyalGreen}{rgb}{0.0236,0.6179,0.0894}
\definecolor{RoyalRed}{rgb}{0.6179,0.0236,0.0894}
\definecolor{LightBlue}{rgb}{0.8544,0.9511,1.0000}
\definecolor{Black}{rgb}{0.0,0.0,0.0}

\definecolor{linkColor}{rgb}{0.0,0.0,0.554}
\definecolor{citeColor}{rgb}{0.0,0.0,0.554}
\definecolor{fileColor}{rgb}{0.0,0.0,0.554}
\definecolor{urlColor}{rgb}{0.0,0.0,0.554}
\definecolor{promptColor}{rgb}{0.0,0.0,0.589}
\definecolor{brkpromptColor}{rgb}{0.589,0.0,0.0}
\definecolor{gapinputColor}{rgb}{0.589,0.0,0.0}
\definecolor{gapoutputColor}{rgb}{0.0,0.0,0.0}

%%  for a long time these were red and blue by default,
%%  now black, but keep variables to overwrite
\definecolor{FuncColor}{rgb}{0.0,0.0,0.0}
%% strange name because of pdflatex bug:
\definecolor{Chapter }{rgb}{0.0,0.0,0.0}
\definecolor{DarkOlive}{rgb}{0.1047,0.2412,0.0064}


\usepackage{fancyvrb}

\usepackage{mathptmx,helvet}
\usepackage[T1]{fontenc}
\usepackage{textcomp}


\usepackage[
            pdftex=true,
            bookmarks=true,        
            a4paper=true,
            pdftitle={Written with GAPDoc},
            pdfcreator={LaTeX with hyperref package / GAPDoc},
            colorlinks=true,
            backref=page,
            breaklinks=true,
            linkcolor=linkColor,
            citecolor=citeColor,
            filecolor=fileColor,
            urlcolor=urlColor,
            pdfpagemode={UseNone}, 
           ]{hyperref}

\newcommand{\maintitlesize}{\fontsize{50}{55}\selectfont}

% write page numbers to a .pnr log file for online help
\newwrite\pagenrlog
\immediate\openout\pagenrlog =\jobname.pnr
\immediate\write\pagenrlog{PAGENRS := [}
\newcommand{\logpage}[1]{\protect\write\pagenrlog{#1, \thepage,}}
%% were never documented, give conflicts with some additional packages

\newcommand{\GAP}{\textsf{GAP}}

%% nicer description environments, allows long labels
\usepackage{enumitem}
\setdescription{style=nextline}

%% depth of toc
\setcounter{tocdepth}{1}





%% command for ColorPrompt style examples
\newcommand{\gapprompt}[1]{\color{promptColor}{\bfseries #1}}
\newcommand{\gapbrkprompt}[1]{\color{brkpromptColor}{\bfseries #1}}
\newcommand{\gapinput}[1]{\color{gapinputColor}{#1}}


\begin{document}

\logpage{[ 0, 0, 0 ]}
\begin{titlepage}
\mbox{}\vfill

\begin{center}{\maintitlesize \textbf{ corefreesub \mbox{}}}\\
\vfill

\hypersetup{pdftitle= corefreesub }
\markright{\scriptsize \mbox{}\hfill  corefreesub  \hfill\mbox{}}
{\Huge \textbf{ A \textsf{GAP} Package for calculating the core-free subgroups and their faithful transitive
permutation representations \mbox{}}}\\
\vfill

{\Huge  0.1 \mbox{}}\\[1cm]
{ 8 November 2022 \mbox{}}\\[1cm]
\mbox{}\\[2cm]
{\Large \textbf{ Claudio Alexandre Piedade\\
   \mbox{}}}\\
{\Large \textbf{ Manuel Delgado\\
  \mbox{}}}\\
\hypersetup{pdfauthor= Claudio Alexandre Piedade\\
   ;  Manuel Delgado\\
  }
\end{center}\vfill

\mbox{}\\
{\mbox{}\\
\small \noindent \textbf{ Claudio Alexandre Piedade\\
   }  Email: \href{mailto://claudioalexandrepiedade@gmail.com} {\texttt{claudioalexandrepiedade@gmail.com}}\\
  Homepage: \href{https://www.fc.up.pt/pessoas/claudio.piedade/} {\texttt{https://www.fc.up.pt/pessoas/claudio.piedade/}}}\\
{\mbox{}\\
\small \noindent \textbf{ Manuel Delgado\\
  }  Email: \href{mailto://} {\texttt{}}}\\
\end{titlepage}

\newpage\setcounter{page}{2}
\newpage

\def\contentsname{Contents\logpage{[ 0, 0, 1 ]}}

\tableofcontents
\newpage

     
\chapter{\textcolor{Chapter }{Introduction}}\label{Chapter_Introduction}
\logpage{[ 1, 0, 0 ]}
\hyperdef{L}{X7DFB63A97E67C0A1}{}
{
  

 corefreesub is a package which calculates the core-free subgroups and their
faithful transitive permutation representations 

 }

   
\chapter{\textcolor{Chapter }{Core Free Subgroups}}\label{Chapter_Core_Free_Subgroups}
\logpage{[ 2, 0, 0 ]}
\hyperdef{L}{X862FB71F7CF27B49}{}
{
  
\section{\textcolor{Chapter }{IsCoreFree}}\label{Chapter_Core_Free_Subgroups_Section_IsCoreFree}
\logpage{[ 2, 1, 0 ]}
\hyperdef{L}{X7A9042C27F0BA85B}{}
{
  

 

\subsection{\textcolor{Chapter }{IsCoreFree}}
\logpage{[ 2, 1, 1 ]}\nobreak
\hyperdef{L}{X7A9042C27F0BA85B}{}
{\noindent\textcolor{FuncColor}{$\triangleright$\enspace\texttt{IsCoreFree({\mdseries\slshape G, H})\index{IsCoreFree@\texttt{IsCoreFree}}
\label{IsCoreFree}
}\hfill{\scriptsize (function)}}\\
\textbf{\indent Returns:\ }
a boolean 



 Returns whether the subgroup \mbox{\texttt{\mdseries\slshape H}} is core-free in its parent group \mbox{\texttt{\mdseries\slshape G}} 
\begin{Verbatim}[commandchars=!@|,fontsize=\small,frame=single,label=Example]
  !gapprompt@gap>| !gapinput@LoadPackage("CoreFreeSub");|
  !gapprompt@gap>| !gapinput@G := SymmetricGroup(4);; H := Subgroup(G, [(1,3)(2,4)]);;|
  !gapprompt@gap>| !gapinput@Core(G,H);|
  Group(())
  !gapprompt@gap>| !gapinput@IsCoreFree(G,H);|
  true
  !gapprompt@gap>| !gapinput@H := Subgroup(G, [(1,4)(2,3), (1,3)(2,4)]);;|
  !gapprompt@gap>| !gapinput@IsCoreFree(G,H);|
  false
  !gapprompt@gap>| !gapinput@Core(G,H); # H is a normal subgroup of G, hence it does not have a trivial core|
  Group([ (1,4)(2,3), (1,3)(2,4) ])
\end{Verbatim}
 }

 }

 
\section{\textcolor{Chapter }{CoreFreeConjugacyClassesSubgroups}}\label{Chapter_Core_Free_Subgroups_Section_CoreFreeConjugacyClassesSubgroups}
\logpage{[ 2, 2, 0 ]}
\hyperdef{L}{X7D0391668782765C}{}
{
  

 

\subsection{\textcolor{Chapter }{CoreFreeConjugacyClassesSubgroups}}
\logpage{[ 2, 2, 1 ]}\nobreak
\hyperdef{L}{X7D0391668782765C}{}
{\noindent\textcolor{FuncColor}{$\triangleright$\enspace\texttt{CoreFreeConjugacyClassesSubgroups({\mdseries\slshape G})\index{CoreFreeConjugacyClassesSubgroups@\texttt{CoreFreeConjugacyClassesSubgroups}}
\label{CoreFreeConjugacyClassesSubgroups}
}\hfill{\scriptsize (function)}}\\
\textbf{\indent Returns:\ }
a list 



 Returns a list of all conjugacy classes of core-free subgroups of \mbox{\texttt{\mdseries\slshape G}} 
\begin{Verbatim}[commandchars=!@|,fontsize=\small,frame=single,label=Example]
  !gapprompt@gap>| !gapinput@LoadPackage("CoreFreeSub");|
  !gapprompt@gap>| !gapinput@G := SymmetricGroup(4);; dh := DihedralGroup(10);;|
  !gapprompt@gap>| !gapinput@CoreFreeConjugacyClassesSubgroups(G);|
  [ Group( () )^G, Group( [ (1,3)(2,4) ] )^G, Group( [ (3,4) ] )^G, Group( [ (2,4,3) ] )^G, Group( [ (1,4)(2,3), (1,3)(2,4) ] )^G, Group( [ (3,4), (1,2)(3,4) ] )^G, Group( [ (1,3,2,4), (1,2)(3,4) ] )^G, Group( [ (3,4), (2,4,3) ] )^G, Group( [ (1,4)(2,3), (1,3)(2,4), (3,4) ] )^G, Group( [ (1,4)(2,3), (1,3)(2,4), (2,4,3) ] )^G, Group( [ (1,4)(2,3), (1,3)(2,4), (2,4,3), (3,4) ] )^G ]
  !gapprompt@gap>| !gapinput@CoreFreeConjugacyClassesSubgroups(dh);|
  [ Group( <identity> of ... )^G, Group( [ f1 ] )^G ] 
\end{Verbatim}
 }

 }

 
\section{\textcolor{Chapter }{AllCoreFreeSubgroups}}\label{Chapter_Core_Free_Subgroups_Section_AllCoreFreeSubgroups}
\logpage{[ 2, 3, 0 ]}
\hyperdef{L}{X821F9BD687383C1F}{}
{
  

 

\subsection{\textcolor{Chapter }{AllCoreFreeSubgroups}}
\logpage{[ 2, 3, 1 ]}\nobreak
\hyperdef{L}{X821F9BD687383C1F}{}
{\noindent\textcolor{FuncColor}{$\triangleright$\enspace\texttt{AllCoreFreeSubgroups({\mdseries\slshape G})\index{AllCoreFreeSubgroups@\texttt{AllCoreFreeSubgroups}}
\label{AllCoreFreeSubgroups}
}\hfill{\scriptsize (function)}}\\
\textbf{\indent Returns:\ }
a list 



 Returns a list of all core-free subgroups of \mbox{\texttt{\mdseries\slshape G}} 
\begin{Verbatim}[commandchars=!@|,fontsize=\small,frame=single,label=Example]
  !gapprompt@gap>| !gapinput@LoadPackage("CoreFreeSub");|
  !gapprompt@gap>| !gapinput@G := SymmetricGroup(4);; dh := DihedralGroup(10);;|
  !gapprompt@gap>| !gapinput@AllCoreFreeSubgroups(G);|
  [ Group(()), Group([ (1,3)(2,4) ]), Group([ (1,4)(2,3) ]), Group([ (1,2)(3,4) ]), Group([ (3,4) ]), Group([ (2,4) ]), Group([ (2,3) ]), Group([ (1,4) ]), Group([ (1,3) ]), Group([ (1,2) ]), Group([ (2,4,3) ]), Group([ (1,3,2) ]), Group([ (1,3,4) ]), Group([ (1,4,2) ]), Group([ (3,4), (1,2)(3,4) ]), Group([ (2,4), (1,3)(2,4) ]), Group([ (2,3), (1,4)(2,3) ]), Group([ (1,3,2,4), (1,2)(3,4) ]), Group([ (1,2,3,4), (1,3)(2,4) ]), Group([ (1,2,4,3), (1,4)(2,3) ]), Group([ (3,4), (2,4,3) ]), Group([ (1,3), (1,3,2) ]), Group([ (1,3), (1,3,4) ]), Group([ (1,4), (1,4,2) ]) ]
  !gapprompt@gap>| !gapinput@AllCoreFreeSubgroups(dh);|
  [ Group([  ]), Group([ f1 ]), Group([ f1*f2^2 ]), Group([ f1*f2^4 ]), Group([ f1*f2 ]), Group([ f1*f2^3 ]) ]
\end{Verbatim}
 }

 }

 
\section{\textcolor{Chapter }{CoreFreeDegrees}}\label{Chapter_Core_Free_Subgroups_Section_CoreFreeDegrees}
\logpage{[ 2, 4, 0 ]}
\hyperdef{L}{X844F04457C6EDC67}{}
{
  

 

\subsection{\textcolor{Chapter }{CoreFreeDegrees}}
\logpage{[ 2, 4, 1 ]}\nobreak
\hyperdef{L}{X844F04457C6EDC67}{}
{\noindent\textcolor{FuncColor}{$\triangleright$\enspace\texttt{CoreFreeDegrees({\mdseries\slshape G})\index{CoreFreeDegrees@\texttt{CoreFreeDegrees}}
\label{CoreFreeDegrees}
}\hfill{\scriptsize (function)}}\\
\textbf{\indent Returns:\ }
a list 



 Returns a list of all possible degrees of faithful transitive permutation
representations of \mbox{\texttt{\mdseries\slshape G}}. The degrees of a faithful transitive permutation representation of \mbox{\texttt{\mdseries\slshape G}} are the index of its core-free subgroups. 
\begin{Verbatim}[commandchars=!@|,fontsize=\small,frame=single,label=Example]
  !gapprompt@gap>| !gapinput@LoadPackage("CoreFreeSub");|
  !gapprompt@gap>| !gapinput@G := SymmetricGroup(4);; dh := DihedralGroup(10);;|
  !gapprompt@gap>| !gapinput@CoreFreeDegrees(G);|
  [ 24, 12, 8, 6, 4 ]
  !gapprompt@gap>| !gapinput@CoreFreeDegrees(dh);|
  [10, 5]
\end{Verbatim}
 }

 }

 }

 \def\indexname{Index\logpage{[ "Ind", 0, 0 ]}
\hyperdef{L}{X83A0356F839C696F}{}
}

\cleardoublepage
\phantomsection
\addcontentsline{toc}{chapter}{Index}


\printindex

\immediate\write\pagenrlog{["Ind", 0, 0], \arabic{page},}
\newpage
\immediate\write\pagenrlog{["End"], \arabic{page}];}
\immediate\closeout\pagenrlog
\end{document}
